%% Generated by Sphinx.
\def\sphinxdocclass{report}
\documentclass[letterpaper,10pt,english]{sphinxmanual}
\ifdefined\pdfpxdimen
   \let\sphinxpxdimen\pdfpxdimen\else\newdimen\sphinxpxdimen
\fi \sphinxpxdimen=.75bp\relax

\PassOptionsToPackage{warn}{textcomp}
\usepackage[utf8]{inputenc}
\ifdefined\DeclareUnicodeCharacter
% support both utf8 and utf8x syntaxes
\edef\sphinxdqmaybe{\ifdefined\DeclareUnicodeCharacterAsOptional\string"\fi}
  \DeclareUnicodeCharacter{\sphinxdqmaybe00A0}{\nobreakspace}
  \DeclareUnicodeCharacter{\sphinxdqmaybe2500}{\sphinxunichar{2500}}
  \DeclareUnicodeCharacter{\sphinxdqmaybe2502}{\sphinxunichar{2502}}
  \DeclareUnicodeCharacter{\sphinxdqmaybe2514}{\sphinxunichar{2514}}
  \DeclareUnicodeCharacter{\sphinxdqmaybe251C}{\sphinxunichar{251C}}
  \DeclareUnicodeCharacter{\sphinxdqmaybe2572}{\textbackslash}
\fi
\usepackage{cmap}
\usepackage[T1]{fontenc}
\usepackage{amsmath,amssymb,amstext}
\usepackage{babel}
\usepackage{times}
\usepackage[Bjarne]{fncychap}
\usepackage{sphinx}

\fvset{fontsize=\small}
\usepackage{geometry}

% Include hyperref last.
\usepackage{hyperref}
% Fix anchor placement for figures with captions.
\usepackage{hypcap}% it must be loaded after hyperref.
% Set up styles of URL: it should be placed after hyperref.
\urlstyle{same}
\addto\captionsenglish{\renewcommand{\contentsname}{Contents:}}

\addto\captionsenglish{\renewcommand{\figurename}{Fig.}}
\addto\captionsenglish{\renewcommand{\tablename}{Table}}
\addto\captionsenglish{\renewcommand{\literalblockname}{Listing}}

\addto\captionsenglish{\renewcommand{\literalblockcontinuedname}{continued from previous page}}
\addto\captionsenglish{\renewcommand{\literalblockcontinuesname}{continues on next page}}
\addto\captionsenglish{\renewcommand{\sphinxnonalphabeticalgroupname}{Non-alphabetical}}
\addto\captionsenglish{\renewcommand{\sphinxsymbolsname}{Symbols}}
\addto\captionsenglish{\renewcommand{\sphinxnumbersname}{Numbers}}

\addto\extrasenglish{\def\pageautorefname{page}}

\setcounter{tocdepth}{2}



\title{K-CHAIN Documentation}
\date{Mar 31, 2019}
\release{}
\author{Nurali Virani}
\newcommand{\sphinxlogo}{\vbox{}}
\renewcommand{\releasename}{}
\makeindex
\begin{document}

\pagestyle{empty}
\maketitle
\pagestyle{plain}
\sphinxtableofcontents
\pagestyle{normal}
\phantomsection\label{\detokenize{index::doc}}


DARPA ASKE TA1


\chapter{Module: kChain}
\label{\detokenize{index:module-kChain}}\label{\detokenize{index:module-kchain}}\index{kChain (module)@\spxentry{kChain}\spxextra{module}}
This module consists of kChainModel class to create, fit, append, and update
K-CHAIN models in TensorFlow
\index{kChainModel (class in kChain)@\spxentry{kChainModel}\spxextra{class in kChain}}

\begin{fulllineitems}
\phantomsection\label{\detokenize{index:kChain.kChainModel}}\pysiglinewithargsret{\sphinxbfcode{\sphinxupquote{class }}\sphinxcode{\sphinxupquote{kChain.}}\sphinxbfcode{\sphinxupquote{kChainModel}}}{\emph{debug=False}}{}~\index{\_\_init\_\_() (kChain.kChainModel method)@\spxentry{\_\_init\_\_()}\spxextra{kChain.kChainModel method}}

\begin{fulllineitems}
\phantomsection\label{\detokenize{index:kChain.kChainModel.__init__}}\pysiglinewithargsret{\sphinxbfcode{\sphinxupquote{\_\_init\_\_}}}{\emph{debug=False}}{}
Initialize object of type K-CHAIN model.
\begin{quote}\begin{description}
\item[{Parameters}] \leavevmode
\sphinxstyleliteralstrong{\sphinxupquote{debug}} (\sphinxhref{https://docs.python.org/3/library/functions.html\#bool}{\sphinxstyleliteralemphasis{\sphinxupquote{bool}}}) \textendash{} various print statements throughout the code
execution will be executed to help in debugging.

\end{description}\end{quote}

\end{fulllineitems}

\index{\_createEqnModel() (kChain.kChainModel method)@\spxentry{\_createEqnModel()}\spxextra{kChain.kChainModel method}}

\begin{fulllineitems}
\phantomsection\label{\detokenize{index:kChain.kChainModel._createEqnModel}}\pysiglinewithargsret{\sphinxbfcode{\sphinxupquote{\_createEqnModel}}}{\emph{inputVar}, \emph{outputVar}, \emph{mdlName}, \emph{eqMdl}}{}
Build a K-CHAIN model using input and output variables from the KG and
the physics equation.
\begin{quote}\begin{description}
\item[{Parameters}] \leavevmode\begin{itemize}
\item {} 
\sphinxstyleliteralstrong{\sphinxupquote{inputVar}} (\sphinxstyleliteralemphasis{\sphinxupquote{JSON array}}) \textendash{} array of JSON variable objects with name, type, and value fields

\item {} 
\sphinxstyleliteralstrong{\sphinxupquote{outputVar}} (\sphinxstyleliteralemphasis{\sphinxupquote{JSON array}}) \textendash{} array of JSON variable objects with name, type, and value fields

\item {} 
\sphinxstyleliteralstrong{\sphinxupquote{mdlName}} (\sphinxstyleliteralemphasis{\sphinxupquote{string}}) \textendash{} Name to assign to the final model (E.g.: ‘Newtons2ndLaw’)

\item {} 
\sphinxstyleliteralstrong{\sphinxupquote{eqMdl}} (\sphinxstyleliteralemphasis{\sphinxupquote{string}}) \textendash{} Equation relating inputs to output (E.g.: “c = a * b”)

\end{itemize}

\item[{Returns}] \leavevmode
Computational graph of the physics equation
metagraphLoc (string): Location on disk where computational model was stored

\item[{Return type}] \leavevmode
mdl (TensorFlow Graph)

\end{description}\end{quote}

\end{fulllineitems}

\index{\_createNNModel() (kChain.kChainModel method)@\spxentry{\_createNNModel()}\spxextra{kChain.kChainModel method}}

\begin{fulllineitems}
\phantomsection\label{\detokenize{index:kChain.kChainModel._createNNModel}}\pysiglinewithargsret{\sphinxbfcode{\sphinxupquote{\_createNNModel}}}{\emph{inputVar}, \emph{outputVar}, \emph{mdlName}}{}
Build a K-CHAIN model as a neural network using input and output
variables from the KG.
\begin{quote}\begin{description}
\item[{Parameters}] \leavevmode\begin{itemize}
\item {} 
\sphinxstyleliteralstrong{\sphinxupquote{inputVar}} (\sphinxstyleliteralemphasis{\sphinxupquote{JSON array}}) \textendash{} array of JSON variable objects with name (as in dataset) and type fields

\item {} 
\sphinxstyleliteralstrong{\sphinxupquote{outputVar}} (\sphinxstyleliteralemphasis{\sphinxupquote{JSON array}}) \textendash{} array of JSON variable objects with name (as in dataset) and type fields

\item {} 
\sphinxstyleliteralstrong{\sphinxupquote{mdlName}} (\sphinxstyleliteralemphasis{\sphinxupquote{string}}) \textendash{} Name to assign to the final model (E.g.: ‘Newtons2ndLaw’)

\end{itemize}

\item[{Returns}] \leavevmode

computational graph of the neural network
metagraphLoc (string):
\begin{quote}

Location on disk where computational model is stored
\end{quote}


\item[{Return type}] \leavevmode
mdl (TensorFlow Graph)

\end{description}\end{quote}

\end{fulllineitems}

\index{\_getVarType() (kChain.kChainModel method)@\spxentry{\_getVarType()}\spxextra{kChain.kChainModel method}}

\begin{fulllineitems}
\phantomsection\label{\detokenize{index:kChain.kChainModel._getVarType}}\pysiglinewithargsret{\sphinxbfcode{\sphinxupquote{\_getVarType}}}{\emph{typeStr}}{}
Obtain tensorflow datatypes for variable type information from KG
\begin{quote}\begin{description}
\item[{Parameters}] \leavevmode
\sphinxstyleliteralstrong{\sphinxupquote{typeStr}} (\sphinxstyleliteralemphasis{\sphinxupquote{string}}) \textendash{} String denoting type of variable with possible values of bool,
integer, float, and double (default).

\item[{Returns}] \leavevmode
datatype in TensorFlow   (e.g. tf.bool)

\end{description}\end{quote}

\end{fulllineitems}

\index{\_makePyFile() (kChain.kChainModel method)@\spxentry{\_makePyFile()}\spxextra{kChain.kChainModel method}}

\begin{fulllineitems}
\phantomsection\label{\detokenize{index:kChain.kChainModel._makePyFile}}\pysiglinewithargsret{\sphinxbfcode{\sphinxupquote{\_makePyFile}}}{\emph{stringfun}}{}
Write the formatted code into a python module for conversion to tensorflow graph
\begin{quote}\begin{description}
\item[{Parameters}] \leavevmode
\sphinxstyleliteralstrong{\sphinxupquote{stringfun}} (\sphinxstyleliteralemphasis{\sphinxupquote{string}}) \textendash{} formatted python code as string to be written in python file

\end{description}\end{quote}

\end{fulllineitems}

\index{build() (kChain.kChainModel method)@\spxentry{build()}\spxextra{kChain.kChainModel method}}

\begin{fulllineitems}
\phantomsection\label{\detokenize{index:kChain.kChainModel.build}}\pysiglinewithargsret{\sphinxbfcode{\sphinxupquote{build}}}{\emph{inputVar}, \emph{outputVar}, \emph{mdlName}, \emph{dataLoc=None}, \emph{eqMdl=None}}{}
Build a K-CHAIN model using input and output variables from the KG.
\begin{quote}\begin{description}
\item[{Parameters}] \leavevmode\begin{itemize}
\item {} 
\sphinxstyleliteralstrong{\sphinxupquote{inputVar}} (\sphinxstyleliteralemphasis{\sphinxupquote{JSON array}}) \textendash{} array of JSON variable objects with name (as in dataset), type, and value fields

\item {} 
\sphinxstyleliteralstrong{\sphinxupquote{outputVar}} (\sphinxstyleliteralemphasis{\sphinxupquote{JSON array}}) \textendash{} array of JSON variable objects with name (as in dataset), type, and value fields

\item {} 
\sphinxstyleliteralstrong{\sphinxupquote{mdlName}} (\sphinxstyleliteralemphasis{\sphinxupquote{string}}) \textendash{} Name to assign to the final model (E.g.: ‘Newtons2ndLaw’)

\item {} 
\sphinxstyleliteralstrong{\sphinxupquote{dataLoc}} (\sphinxstyleliteralemphasis{\sphinxupquote{string}}) \textendash{} Location of dataset as .csv with Row 1 - Variables names,
Row 2 - Units, Row 3 onwards - data (default = None)

\item {} 
\sphinxstyleliteralstrong{\sphinxupquote{eqMdl}} (\sphinxstyleliteralemphasis{\sphinxupquote{string}}) \textendash{} 

\end{itemize}

\end{description}\end{quote}

\end{fulllineitems}

\index{evaluate() (kChain.kChainModel method)@\spxentry{evaluate()}\spxextra{kChain.kChainModel method}}

\begin{fulllineitems}
\phantomsection\label{\detokenize{index:kChain.kChainModel.evaluate}}\pysiglinewithargsret{\sphinxbfcode{\sphinxupquote{evaluate}}}{\emph{inputVar}, \emph{outputVar}, \emph{mdlName}}{}
Evaluates a model with given inputs to compute output values
\begin{quote}\begin{description}
\item[{Parameters}] \leavevmode\begin{itemize}
\item {} 
\sphinxstyleliteralstrong{\sphinxupquote{inputVar}} (\sphinxstyleliteralemphasis{\sphinxupquote{JSON array}}) \textendash{} array of JSON variable objects with name, type, and value fields

\item {} 
\sphinxstyleliteralstrong{\sphinxupquote{outputVar}} (\sphinxstyleliteralemphasis{\sphinxupquote{JSON array}}) \textendash{} array of JSON variable objects with name, type, and value fields

\item {} 
\sphinxstyleliteralstrong{\sphinxupquote{mdlName}} (\sphinxstyleliteralemphasis{\sphinxupquote{string}}) \textendash{} Name to model to use (E.g.: ‘Newtons2ndLaw’)

\end{itemize}

\item[{Returns}] \leavevmode
array of JSON variable objects with name, type, and value fields.
The resulting output of the computation is assigned to the value
field of the JSON object.

\item[{Return type}] \leavevmode
outputVar (JSON array)

\end{description}\end{quote}

\end{fulllineitems}

\index{fitModel() (kChain.kChainModel method)@\spxentry{fitModel()}\spxextra{kChain.kChainModel method}}

\begin{fulllineitems}
\phantomsection\label{\detokenize{index:kChain.kChainModel.fitModel}}\pysiglinewithargsret{\sphinxbfcode{\sphinxupquote{fitModel}}}{\emph{dataset}, \emph{inputVar}, \emph{outputVar}, \emph{mdlName}}{}
Fit a K-CHAIN model using input and output variables from the KG and
the corresponding dataset.
\begin{quote}\begin{description}
\item[{Parameters}] \leavevmode\begin{itemize}
\item {} 
\sphinxstyleliteralstrong{\sphinxupquote{dataset}} (\sphinxstyleliteralemphasis{\sphinxupquote{Pandas Dataframe}}) \textendash{} dataset with inputs and outputs

\item {} 
\sphinxstyleliteralstrong{\sphinxupquote{inputVar}} (\sphinxstyleliteralemphasis{\sphinxupquote{JSON array}}) \textendash{} array of JSON variable objects with name (as in dataset) and type fields

\item {} 
\sphinxstyleliteralstrong{\sphinxupquote{outputVar}} (\sphinxstyleliteralemphasis{\sphinxupquote{JSON array}}) \textendash{} array of JSON variable objects with name (as in dataset) and type fields

\item {} 
\sphinxstyleliteralstrong{\sphinxupquote{mdlName}} (\sphinxstyleliteralemphasis{\sphinxupquote{string}}) \textendash{} Name to assign to the final model (E.g.: ‘Newtons2ndLaw’)

\end{itemize}

\item[{Returns}] \leavevmode
Location on disk where computational model and trained parameters are stored

\item[{Return type}] \leavevmode
metagraphLoc (string)

\end{description}\end{quote}

\end{fulllineitems}

\index{getDataset() (kChain.kChainModel method)@\spxentry{getDataset()}\spxextra{kChain.kChainModel method}}

\begin{fulllineitems}
\phantomsection\label{\detokenize{index:kChain.kChainModel.getDataset}}\pysiglinewithargsret{\sphinxbfcode{\sphinxupquote{getDataset}}}{\emph{dataLoc=None}}{}
Create Pandas DataFrame from identified csv.
\begin{quote}\begin{description}
\item[{Parameters}] \leavevmode
\sphinxstyleliteralstrong{\sphinxupquote{dataLoc}} (\sphinxstyleliteralemphasis{\sphinxupquote{string}}) \textendash{} Location of dataset as .csv with Row 1 - Variables names,
Row 2 - Units, Row 3 onwards - data (default = None)

\item[{Returns}] \leavevmode
DataFrame with values read from csv file

\item[{Return type}] \leavevmode
df (Pandas DataFrame)

\end{description}\end{quote}

\end{fulllineitems}


\end{fulllineitems}



\chapter{Indices and tables}
\label{\detokenize{index:indices-and-tables}}\begin{itemize}
\item {} 
\DUrole{xref,std,std-ref}{genindex}

\item {} 
\DUrole{xref,std,std-ref}{modindex}

\item {} 
\DUrole{xref,std,std-ref}{search}

\end{itemize}


\renewcommand{\indexname}{Python Module Index}
\begin{sphinxtheindex}
\let\bigletter\sphinxstyleindexlettergroup
\bigletter{k}
\item\relax\sphinxstyleindexentry{kChain}\sphinxstyleindexpageref{index:\detokenize{module-kChain}}
\end{sphinxtheindex}

\renewcommand{\indexname}{Index}
\printindex
\end{document}